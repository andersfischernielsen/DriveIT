\section{Object Design}
\subsection{WindowsClient Controllers}
\subsubsection{Reuse \& Encapsulation}
To perform the Create, Read, Update and Delete functionality The \texttt{DriveIT Windows Client} uses a number of controllers. These controllers each handle the different kinds of entities which the system supports. To Reuse code and therefore making it easier to test, maintain and makes the system less prone to erros, the class \texttt{DriveITWebAPI} is created. This class encapsulate all http-request code to contact the API and uses generics to allow all the controllers to reuse the methods. 
Therefore all the entity controllers know nothing of the \texttt{DriveIT Web API} or the code to contact it, and with a few adjustments the \texttt{DriveITWebAPI} could be reused in any other project involving a REST based web API.

\subsection{Persistent Storage}

\subsubsection{Reuse}
The \texttt{EntitiyStorage} class is very specific for the \texttt{DriveIT System} since it only deals with creating, reading, updating and deleting \texttt{DriveIT} entities. It therefore does not provide any major re-usability for other purposes than this. 

Adding re-usability could be accomplished by using generics (supported by the C\# language), which would allow the system to create, read, update and delete any input entity, provided it was supported by the given \texttt{DbContext}. This would allow for greater extensibility and easier code maintenance.

\subsubsection{Inheritance}
The \texttt{DriveITContext} extends an \texttt{IdentityDbContext} which provides the built in functionality for handling user logins and different user roles, which is used for the login- and access control functionality of the \texttt{DriveIT System}.
The \texttt{IdentityDbContext} in turn extends a \texttt{DbContext} which enables \texttt{Entity Framework} to save the User entities (along with every other \texttt{DriveIT} entity) in the underlying Microsoft SQL database.

\subsubsection{Encapsulation}
Encapsulation has been a high priority during development of the \texttt{DriveIT System}. 

Keeping functionality of different sub systems encapsulated has helped debugging and refactoring. Extensibility is also made easier in the future.

Encapsulation has been used in the \texttt{Persistent Storage} system in methods dealing with retrieving, editing and deleting entities. Methods generally have few side effects and only do not keep references to others classes after performing their task.

A design choice in implementing the methods were to only keep a \texttt{DriveITContext} (which inherits from \texttt{DbContext}) in memory for a short amount of time to ensure that the context only fulfilled its purpose in dealing with the entities and then was discarded.

Updating entities requires copying all attributes of the entity which is done in separate methods without side effects.