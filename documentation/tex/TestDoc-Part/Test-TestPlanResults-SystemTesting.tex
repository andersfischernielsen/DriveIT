\section{System Testing}
The \texttt{DriveIT System} has been under continuous testing throughout development.

As soon as the working prototype had been developed, the different components of the system were put together and system tested. 
By having a bottom-up approach the system tests were mainly focused on during the end of development after all lower sub systems had been unit- and integration tested.

\subsection{Scope}
The scope of the system testing was primarily focused on the main scenarios and use cases. 
These should be supported, so that users are able to complete their tasks. 

Testing every single area of the system was not in scope, since this would take a very long time which was not available.
Testing to find every error was outside of the scope, since this would take too much time. Only main functionality and use cases were the focus points of the testing.

\subsection{Approach}
The \texttt{DriveIT Web} and \texttt{DriveIT Windows Client} have been tested against the specified use cases in the Requirement Analysis Document to see whether every use case was possible to complete inside the functional and nonfunctional requirements.

At the end of development the near-finished system was system tested by the team. The team sat down and opened a release version of the \texttt{DriveIT System}.

Every requirement from the Requirement Analysis Document was tested to see whether the system fulfills the users' tasks.
Minor errors were found and fixed, and system tests were run again.
When a use case was completed in an acceptable fashion the test was deemed as passed.

\subsection{Requirements}
The systems were tested so that it was possible to complete every use case while staying inside the functional and nonfunctional requirements.

\subsection{Results}
Both the \texttt{DriveIT Web} system and \texttt{DriveIT Windows Client} have been tested in this loose manner. 
Both systems have completed the tests.