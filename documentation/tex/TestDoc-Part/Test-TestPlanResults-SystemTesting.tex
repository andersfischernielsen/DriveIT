\section{System Testing}
The \texttt{DriveIT System} has been under continuous testing throughout development.

As soon as the working prototype had been developed, the different components of the system were put together and system tested. 
By having a bottom-up test approach the system tests were mainly focused on during the end of development after all lower sub systems had been unit and integration tested.

\subsection{Scope}
The scope of the system tests was primarily focused on the main scenarios and use cases. 
These should be supported so that users are able to complete their tasks. 

Testing every single area of the system was not in scope, since this would take a very long time that was not available.
Main functionality and use cases were the focus points of the testing.

\subsection{Approach}
The \texttt{DriveIT Web Client} and \texttt{DriveIT Windows Client} have been tested against the specified use cases in the Requirement Analysis Document to see whether every the use cases are possible to complete inside the functional and non-functional requirements.

At the end of development the near-finished \texttt{DriveIT System} was system tested by the developers. The developers sat down and opened a release version of the \texttt{DriveIT System}.

Every requirement from the Requirement Analysis Document was tested to see whether the system fulfills the functionality specified.
Minor errors were found and fixed, e.g. long image path strings would not be saved, and system tests were run again.\\
When a use case was completed in an acceptable fashion the test was deemed as passed.

\subsection{Requirements}
The systems were tested so that it was possible to complete every use case while staying inside the functional and non-functional requirements.

\subsection{Results}
Both the \texttt{DriveIT Web Client} and \texttt{DriveIT Windows Client} have passed the tests.