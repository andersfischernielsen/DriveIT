\section{Acceptance Testing}
Since the project has no real product owner, acceptance testing was done at each \textit{SCRUM} sprint by the team members. At the end of the last sprint, \textit{ref. \ref{sec:sprint4}}, two acceptance tests were made, one for a potential employee of the used car dealership who used the \texttt{DriveIT Web Client} and the \texttt{DriveIT Windows Client}, and another for a potential customer who used the \texttt{DriveIT Web Client}.

\subsection{Employee - Acceptance Testing}
The acceptance testing for employees are based on the scenarios: \texttt{Sell a Used Car to a Customer} \textit{\ref{sec:scenario-sellcar}}, \texttt{Put a New Used Car Up for Sale} \textit{\ref{sec:scenario-createcar}}. To test these scenarios a bachelor student of Digital Media and Design at the IT University of Copenhagen was found. This student is not the perfect representative of a normal car salesman, but the feedback still provides some valid information.
\subsubsection{Feedback}
A couple of the notes taken from the session can be seen below:
\begin{itemize}
	\item The connection between the website and \texttt{DriveIT Windows Client} works fine.
	\item When creating a new entity, the message indicating that the next time you press the "Create/Update" button suddenly \textit{updates} the entity is too subtle.
	\item Data samples of the different properties in the \texttt{EntityViews} could be provided.
	\item When converting a \texttt{ContactRequest} into a \texttt{Sale} it is unclear if the \texttt{ContactRequest} gets deleted or if the \texttt{Employee} must do it himself.
\end{itemize}
Overall the acceptance testing was successful and showed that the system works as intended.
\subsection{Customer - Acceptance Testing}
The acceptance testing for customers are based on the scenarios and use case: \texttt{Create-Account-Use-Case} \textit{\ref{create-account-use-case}}, Comment on a Car, Customer Wishes to Find a Car. To test these scenarios we found an bachelor student in Software Development at IT University of Copenhagen. This student is not the perfect representative of an everyday customer, but the feedback can still provide some valid information.
\subsubsection{Feedback}
Here are a couple of the notes we took from the session.
\begin{itemize}
	\item The website is easy to understand.
	\item Units are not provided on properties - this causes some confusion.
	\item Password creation is difficult since password requirements are not provided to begin with.
\end{itemize}
Overall the acceptance testing was successful and showed that the system works as intended to some degree.