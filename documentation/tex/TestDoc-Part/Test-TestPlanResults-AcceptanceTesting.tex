\section{Acceptance Testing}
Since the project has no real product owner, acceptance test was done at each sprint by the team members. At the end of the last sprint (\todo{refDatSection} sprint 4) two acceptance tests were made, one for a potential employee of the company which uses the \texttt{DriveIT Web Client} and the \texttt{DriveIT Windows Client}, and another for a potential customer which would use the \texttt{DriveIT Web Client}.
\subsection{Employee - Acceptance testing}
The acceptance testing for employees are based on the scenarios: \todo{refDatSection} Sell a used car to a customer, put a new used car up for sale. To test these scenarios we found an bachelor student in Digital Media and Design at IT University of Copenhagen. This student is not the perfect representative of a normal Car sales person, but the feedback can still provide some valid information.
\subsubsection{Feedback}
Here are a couple of the notes we took from the session.
\begin{itemize}
	\item The connection between the website and Windows client works fine.
	\item When creating a new entity, the message indicating that the next time you press the "Create/Update" button now updates the entity is too subtle.
	\item Data samples of the different properties in the Entity Views could be provided.
	\item When converting a \texttt{Contact Request} into a \texttt{Sale} it is unclear if the \texttt{Contact Request} gets deleted or if the \texttt{Employee} must do it himself.
\end{itemize}
Overall the acceptance testing was successful.
\subsection{Customer - Acceptance testing}
The acceptance testing for customers are based on the scenarios: \todo{refDatSection} Create User, Comment on a car, Customer Wishes to find a car. To test these scenarios we found an bachelor student in Software Development at IT University of Copenhagen. This student is not the perfect representative of an everyday customer, but the feedback can still provide some valid information.
\subsubsection{Feedback}
Here are a couple of the notes we took from the session.
\begin{itemize}
	\item The website is easy to understand.
	\item Units are not provided on properties - this causes some confusion.
	\item Password creation is difficult since password requirements are not provided to begin with.
\end{itemize}
Overall the acceptance testing was successful.
	