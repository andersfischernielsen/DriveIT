\section{Sprints}
After each sprint the team sat down for a sprint end meeting. We followed the procedure outlined in the \textit{SCRUM} documentation\footnote{http://en.wikipedia.org/wiki/Scrum\_\%28software\_development\%29} on how to carry out the meeting.\\

The main points to consider were \textit{"What Could Be Improved?"}, \textit{"What Went Well?"} and \textit{"Presentation of Work"}.\\

After reviewing the sprint retrospections some issues required more work to fix during the development of the system.

Writing the documentation in LaTeX from the beginning turned out to be harder for some group members, since they did not have prior in depth experience with the syntax. 

Staying on top of updating the documentation while developing the code base of the \texttt{DriveIT System} was harder than first estimated. Some major revisions had to be made to the documentation near the end of the project, since they had not been updated during development.

Finding a fitting amount of work for the team members took some time, especially due to the time team members had available for working on the project exclusively changed over the course of the project.

Using Git as a version control system was easy and fit the team well. Branching was used extensively and helped avoiding overwriting other team members' work. 

Stand-up meetings worked well, and beginning a day's work was made easier by knowing which team members did what. \\

The full summaries of the Sprint End meetings, including the following Sprint retrospections, can be seen in Appendix \ref{sec:scrum-sprints}.
