\section{SCRUM Organizations}

The team had an introductory meeting at the very beginning of the project. \\
This meeting was not a SCRUM meeting per se, but more a meeting to discuss expectiations, when team members were available, who should do what and how much people could work.
The first proper SCRUM meeting was planned and was held a few days later.

Following this meeting the SCRUM process was generally followed. Stand-up meetings were held every morning without waiting for missing team members, sprints were planned and evaluated and tasks were assigned. \\
Using the SCRUM planning tool in Visual Studio Online greatly benefitted the team, since no post its etc. were needed for planning tasks and backlog items, though a whiteboard still came in handy.

The team members each took turns acting as SCRUM master, with no concrete roles.\\
During the project the different members were involved in different areas of work, not only focusing on one aspect of the project. Still, all team members had an area of expertise they.

Sprints were each a week long, began on Tuesdays and ended on Wednesdays.\\
This gave the team the ability to work over weekends, where some members had more time to work. The retrospective meeting and the sprint planning meeting could be dealt with over the course of one day, which had the benefit that members could get straight to work the next day.

Presentation of work and sprint retrospection usually happened in one meeting. This happened quite naturally since discussion of how the work had gone over the course of the sprint usually started after showing off the work.