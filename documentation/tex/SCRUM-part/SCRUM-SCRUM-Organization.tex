\section{SCRUM Organisations}

The team had an introductory meeting at the very beginning of the project. \\
This meeting was not a \textit{SCRUM} meeting, but rather a meeting to make a cooperation agreement, in compliance with the available time and expectations of the team members.
At this meeting the first proper SCRUM meeting was scheduled and took place a few days later.

Subsequent to the first \textit{SCRUM} meeting, the \textit{SCRUM} process was generally followed.
Stand-up meetings took place every morning at the scheduled time. At the sprint start meeting, tasks were assigned, sprints were planned, and at each sprint end meeting, sprints were evaluated and finished work was showcased. \\
Using the \textit{SCRUM} planning tool in \textit{Visual Studio Online} greatly benefit the team, since no post-its etc. were needed for planning tasks and backlog items, though a whiteboard still came in handy.

The team members each took turns acting as SCRUM master.\\
During the project the different members were involved in different areas of work, not only focusing on one aspect of the project. Still, each team member had a main area of focus.

Sprints were each a week long, began on Wednesdays and ended on Tuesdays.\\
This gave the team the ability to work over weekends, where some members had more time to work. The retrospective meetings and the sprint planning meetings could be dealt with over the course of one day, which had the benefit that members could get straight to work the next day.

Presentations of the finished work and sprint retrospection usually happened in one meeting. This happened quite naturally since discussions of how the work had gone over the course of the sprint usually started after showing off the work.
