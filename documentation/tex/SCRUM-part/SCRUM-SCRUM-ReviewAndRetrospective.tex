\section{Review and Retrospective}

Generally the retrospective meetings were over very quickly.\\
The team worked quite well and did not find any major issues during the sprints.\\ 
The main issues found were the work estimates, where some tasks took longer or less time than expected and some team members either overallocating themselves or not allocating enough to themselves.\\
Focusing on the documentation in the first sprint gave the team a solid foundation for the ensuing programming sprints, which was a great benefit.\\
Using SCRUM proved to be very effective in planning the work. Having shorter sprints helped keep spirits up for team members, and ensured that everyone did not drown in work. The team was always able to see the end of the next sprint and how much work was left.
Using an iterative approach was a benefit and the team appreciated the ability to change course during the project. \\
Having a waterfall-like approach would not have given the team the same abilities to do this, and unexpected problems that were discovered during the project would not have been as easy to fix. \\
The fact that the team was relatively small and therefore well-suited for the SCRUM method helped a great deal as well. Had the team been larger other issues might have surfaced, that the waterfall approach would have handled better.