\section{Use Case Models}

\subsection{Scenarios}
Six specific scenarios have been chosen for detailed specification. These are the scenarios that are the most important to get a good understanding of the more complicated features of the system. A list of the scenarios not further specified can be seen in appendix \ref{sec:unspecified-use-case-models}

The scenarios that chosen to focus on are shown below. These scenarios were chosen since they represent some of the paramount parts of the system functionality, thus analysing these further will ensure that the main logic of the system is handled correctly.

\subsubsection{Scenario: Sell a used car to a customer}
\HRule \\[0.4cm]
\textbf{Actors:} \texttt{Bob:Employee, Alice:Customer}\\
\HRule \\[0.4cm]
\textbf{Flow:} \\
\begin{enumerate}
\item Alice has choosen to get contacted by DriveIT for a nice Mercedes Benz she saw on the website
\item Bob sees on his DriveIT client that Alice wants to be contacted about the given car
\item Bob uses the system to get contact info on Alice and some info about the car and gives Alice a call
\item Bob and Alice talks on the phone and Bob succeeds in convincing Alice that she should buy this car
\item Bob uses the system to make an order with Alice and the car and sends the bill.
\end{enumerate}
\HRule \\[0.4cm]

\subsubsection{Scenario: Customer does not pick up. Employee sends e-mail}
\HRule \\[0.4cm]
\textbf{Actors:} \texttt{Robert:Employee, Joe:Customer}\\
\HRule \\[0.4cm]
\textbf{Flow:} \\
\begin{enumerate}
\item Robert sees that Joe wants to be contacted about a BMW.
\item Robert finds the info Joe has registered about himself in order to find his phone number.
\item Robert calls Joe who does not pick up.
\item Robert looks in Joe's info again to find his e-mail address.
\item Robert sends an e-mail to Joe using an external e-mailing system.
\end{enumerate}
\HRule \\[0.4cm]

\subsubsection{Scenario: Customer wishes to find a car}
\HRule \\[0.4cm]
\textbf{Actors:} \texttt{Jane:Customer}\\
\HRule \\[0.4cm]
\textbf{Flow:} \\
\begin{enumerate}
 \item Jane, recently divorced, just had her tricycle stolen. With her regular commute of 36 miles each way, she spontaneously decides to buy a car.
 \item Jane faintly recalls the midnight informercial she viewed the previous night while enjoying a gallon of Ben \& Jerrys and binge-watching Bridget Jones.
 \item After removing the heap of dirty clothes from her chair and finding her laptop, she swiftly types in the URL for DriveIT's webpage.
 \item Upon entering the website, she is prompted with a search page, where she can sort cars by a list of criterias.
 \item She quickly sports the `sort by price' criteria, and proceeds to locate the nearest blunt object to break her piggy bank.
 \item After counting her pennies, nickels and dimes she sets the `sort by price'-field to under 35.23\$.
 \item Jane continues to click the `search' button and recieves the list of 3 cars.
 \item Jane scrolls through the cars, looking at the specifications with her needs in mind, and ends up falling in love with the 1899 Horsey Horseless.
\end{enumerate}
\HRule \\[0.4cm]


\subsubsection{Scenario: Get Contacted By Employee}
\HRule \\[0.4cm]
\textbf{Actors:} \texttt{Alice:Customer, Bob:Employee}\\
\HRule \\[0.4cm]
\textbf{Flow:} \\
\begin{enumerate}
    \item Alice has been browsing the DriveIT site for a while and has found a car that she really likes.
    \item Since Alice is not in a hurry to get contacted, she decides that the wants to get contacted by a DriveIT employee. 
    \item Alice presses the button named "Request to get contacted".
    \item Alice sees that the site where she requested to get contacted now says that a contact request has been sent.
\end{enumerate}
\HRule \\[0.4cm]


\subsubsection{Scenario: Put a New Used Car Up For Sale}
\HRule \\[0.4cm]
\textbf{Actors:} \texttt{Bob:Employee}\\
\HRule \\[0.4cm]
\textbf{Flow:} \\
\begin{enumerate}
\item The car lot has just purchased a new used car.
\item Bob needs to put the newly purchased car up for sale, so that the customers of the car lot can see and hopefully buy it.
\item Bob knows the make and model of the car, but not any other information. He uses the system to find additional information about the car. 
\item Bob has also taken a photo of the car, and uploads this to the page of the car. 
\item Bob has put all the information and photos he knows into the system, and saves the information. 
\item Bob double-checks that the car is saved and put up for sale. He is happy to see that that is the case.
\end{enumerate}
\HRule \\[0.4cm]


\subsubsection{Scenario: Customer comments on a car}
\HRule \\[0.4cm]
\textbf{Actors:} \texttt{John:User}\\
\HRule \\[0.4cm]
\textbf{Flow:} \\
\begin{enumerate}
\item John is browsing for cars, and meanwhile browsing he finds a Toyoto identical to one he once had himself.
\item Since John had been very fond of the car he wants to leave a comment, letting other people know that it is a good car.
\item John is already signed in, and is therefore able to comment right away.
\item He writes the comment and posts it.
\item John is now able to see his comment, and can delete or edit it if he desires to.
\end{enumerate}
\HRule \\[0.4cm]


\subsection{Use Case Model}
The use case model (fig. \ref{fig:UseCaseModel}) shows which actors can perform which use-cases.\\

The unregistered customer can browse cars, see a detailed view of a specific car, and browse employees.\\
A registered customer can do the same as the unregistered customer, but has access to modifying his or her user profile, as well as requesting to get contacted by an employee and the ability to comment on cars that are up for sale.
They also have the ability to see current contact requests that they have made and see all their previous orders.\\

An employee can browse cars, view details about cars, browse employees, modify, create, and view the list of customers who wants to be contacted regarding cars for sale, sell cars, and create, modify, and remove existing cars for sale on the web page.\\

The administrator is a special kind of employee who can access everything an employee can, as well as creating, modifying and removing other employees, and delete customer accounts.\\

\begin{figure}[H]
    \centering
        \includegraphics[scale=0.4]{Figures/UseCase-Model}\\
    \caption{The use case model - \texttt{Customer} inherits the unregistred customers cases, and the \texttt{Administrator} inherits from \texttt{Employee}}
  \label{fig:UseCaseModel}
\end{figure}

\newpage
\section{Use Cases}
Due to time constraints only five use cases are explained in detail. The selected use cases are the most critical for the user requirements and are important to support for the system to fulfil the users' requirements.

\usecase{
    title = {CreateOrder},
    label = uc:createorder,
    description = {The \emph{Employee} creates an order when the \emph{Customer} wants to buy a car.},
%    scope = {},
%    level = {},
    actors = {\emph{Customer}, \emph{Employee}},
%    stakeholders and interests = {},
%    precondition = {},
    preconditions = {
        \item The \emph{Customer} and \emph{Employee} has agreed upon a deal regarding the purchase of a given car.
        \item The \emph{Employee} has the Windows client for the DriveIt system available with Internet connection.
    },
%    postcondition = {},
    postconditions = {
        \item An \texttt{Order} has been successfully created and persisted in the database.
        \item The image belonging to the purchased car is now labelled as \texttt{SOLD}.
    },
    main success scenario = {
        \item The \emph{Employee} opens the `make new order' window.
        \item The \emph{Employee} selects the car that the \emph{Customer} desires to buy and fills in the agreed upon price.
        \item The \emph{Employee} selects the \emph{Customer}\'s profile to get relevant contact information. If the \emph{Customer} has no profile, the \emph{Employee} creates one.
        \item When all the required fields to make a valid \texttt{Order} is filled, the \emph{Employee} clicks the \texttt{CreateOrder} button.
        \item If the car has been sold by another \emph{Employee} in the mean time, the system gives an error message.
    },
%    extensions = {
%        \item 1
%        \item 2
%        \item 3
%    },
%    special requirements = {
%       \item 1
%        \item 2
%       \\
%   },
    frequency of occurrence = Daily,
    %open issues = {},
}

\subsubsection{UC-2 : Create User Account}
\label{create-account-use-case}
\HRule \\[0.4cm]
% Description
\textbf{Description:} The \texttt{Customer} registers on the \texttt{Web Client} \\
\HRule \\[0.4cm]
% Actors
\textbf{Actors:} \texttt{Customer}\\
\HRule \\[0.4cm]
% Preconditions
\textbf{Preconditions:} 
\begin{itemize}
    \item  The \texttt{Web Client} is initialized and the device initialising it is connected to the Internet.
    \item  The \texttt{Customer} is on the landing page of the \texttt{Web Client}.
\end{itemize}
\HRule \\[0.4cm]
% Postconditions
\textbf{Postcondition:}
\begin{itemize}
    \item The \texttt{Customer} has registered an account in the system.
    \item The \texttt{Customer} is able to login to the system.
\end{itemize}
\HRule \\[0.4cm]
% Main success scenarios
\textbf{Main success scenario:}
\begin{enumerate}
    \item  The \texttt{Customer} registers an account for the \texttt{Web Client}.
    \item  The \texttt{Customer} selects ´Register' in the main view of the \texttt{Web Client}.
    \item  The \texttt{Customer} inputs an e-mail address, first name, last name, and a password. He/she then selects the save button to save the account.
    \item  If the input is valid, the account is persisted and the \texttt{Customer} is logged in.
\end{enumerate}
\HRule \\[0.4cm]
% Frequency
\textbf{Frequency:}
Often \\
\HRule \\[0.4cm]



\subsubsection{UC-3 : Create Car}
\label{create-car-use-case}
\HRule \\[0.4cm]
% Description
\textbf{Description:} An \texttt{Employee} uses the \texttt{Windows Client} to create a new car and thereby making it available for sale. \\
\HRule \\[0.4cm]
% Actors
\textbf{Actors:} \texttt{Employee}\\
\HRule \\[0.4cm]
% Preconditions
\textbf{Preconditions:} 
\begin{itemize}
    \item The \texttt{Employee} at least has the manufacturer and model of the car plus at least one picture.
    \item The \texttt{Employee} is signed in to the \texttt{Windows Client} and is connected to the Internet.
\end{itemize}
\HRule \\[0.4cm]
% Postconditions
\textbf{Postcondition:}
\begin{itemize}
    \item The \texttt{Car} is persisted in the database.
    \item The \texttt{Car} is available for sale on the \texttt{Web Client}.
\end{itemize}
\HRule \\[0.4cm]
% Main success scenarios
\textbf{Main success scenario:}
\begin{enumerate}
    \item The \texttt{Employee} signs in to the system.
    \item The \texttt{Employee} chooses the `Car' view and chooses the `Create' button.
    \item The \texttt{Employee} fills out the available data about the car. Given manufactorer and model, the \texttt{Employee} can choose to use the \texttt{Car Query API} to retrieve data about the given car.
    \item The \texttt{Employee} checks that the data is correct and submits the car.
\end{enumerate}
\HRule \\[0.4cm]
% Frequency
\textbf{Frequency:}
A couple of times a week \\
\HRule \\[0.4cm]



\subsubsection{UC-4 : Contact Interested Customer}
\label{contact-interested-customer-use-case}
\HRule \\[0.4cm]
% Description
\textbf{Description:} An \texttt{Employee} contacts a \texttt{Customer} who has requested to get contacted. \\
\HRule \\[0.4cm]
% Actors
\textbf{Actors:} \texttt{Employee, Customer}\\
\HRule \\[0.4cm]
% Preconditions
\textbf{Preconditions:} 
\begin{itemize}
    \item The \texttt{Windows Client} is online and the \texttt{Employee} is signed in to the system.
    \item There exists at least one \texttt{Contact Request} in the system.
\end{itemize}
\HRule \\[0.4cm]
% Postconditions
\textbf{Postcondition:}
\begin{itemize}
    \item The \texttt{Customer} who requested to get contacted has been contacted by the \texttt{Employee}.
\end{itemize}
\HRule \\[0.4cm]
% Main success scenarios
\textbf{Main success scenario:}
\begin{enumerate}
    \item The \texttt{Employee} navigates to the view of \texttt{Contact Requests}.
    \item The \texttt{Employee} chooses the oldest non contacted \texttt{Contact Request}, and open a detailed view of the request.
    \item The \texttt{Employee} contact the \texttt{Customer} through one of the means listed in the detailed view.
    \item Depending on the outcome of the communication between the two, the \texttt{Employee} either deletes the \texttt{Contact Request} or converts it into a \texttt{Sale}.
\end{enumerate}
\HRule \\[0.4cm]
% Frequency
\textbf{Frequency:}
Daily \\
\HRule \\[0.4cm]



\usecase{
    title = {RequestEmployeeContact},
    label = uc:req-emp-contact,
    description = {The \emph{User} wishes to be contacted by an \emph{Employee} about a car.},
%    scope = {},
%    level = {},
    actors = {Initiated by \emph{Customer}},
%    stakeholders and interests = {},
%    precondition = {},
    preconditions = {
    \item The \texttt{DriveIT System} is initialized and online.
    \item The \texttt{Customer} has found a car in the \texttt{DriveIT System} he/she wishes to be contacted about.
    },
%    postcondition = {},    
    postconditions = {
    \item The \emph{Customer} has submitted a request form.,
    \item The request-employee-contact-form has been registered in the \texttt{DriveIT System}.
    },
    main success scenario = {
       \item The \emph{Customer} has found a car he/she wishes to be contacted about.
       \item The \emph{Customer} fills out a request-employee-contact-form and submits it.
       \item The request-employee-contact-form is  registered in the \texttt{DriveIT System}.
       \item Any \emph{Employee} registered in the \texttt{DriveIT System} is now able to view and process the submission.
    },
%    extensions = {
%        \item 1
%        \item 2
%        \item 3
%    },
%    special requirements = {
%	    \item 1
%        \item 2
%		\\
%	},
    frequency of occurrence = Often,
    open issues = {It is unclear what is unclear},
}
