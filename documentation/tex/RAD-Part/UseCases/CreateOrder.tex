% Use case for creating an order when a user has decided to buy a car
% \usepackage{usecases}
\usecase{
    title = {CreateOrder},
    label = uc:create_order,
    description = {
                    The \texttt{Employee} opens the `make new order' window.\\\\
                    The \texttt{Employee} selects the car that the \texttt{Customer} desires to buy and fills in the agreed upon price.\\
                    The \texttt{Employee} selects the \texttt{Customer}\'s profile to get relevant contact information. If the \texttt{Customer} has no profile, the \texttt{Employee} creates one.\\
                    When all the required fields to make a valid \texttt{Order} is filled, the \texttt{Employee} clicks the \texttt{CreateOrder} button.\\
                    If the car has been sold by another \texttt{Employee} in the mean time, the system gives an error message.
                  },
%    scope = {},
%    level = {},
    actors = {\texttt{Customer}, \texttt{Employee}},
%    stakeholders and interests = {},
%    precondition = {},
    preconditions = {
        \item 1 The \texttt{Customer} and \texttt{Employee} has agreed upon a deal regarding the purchase of a given car.
        \item 2 The \texttt{Employee} has the Windows client for the DriveIt system available with Internet connection.
    },
%    postcondition = {},
    postconditions = {
        \item 1 An \texttt{Order} has been successfully created and persisted in the database.
        \item 2 The image belonging to the purchased car is now labelled as \texttt{SOLD}.
    },
    main success scenario = {
        \item The purchase is successful without any errors and the database is successfully updated.
    },
%    extensions = {
%        \item 1
%        \item 2
%        \item 3
%    },
%    special requirements = {
%	    \item 1
%        \item 2
%		\\
%	},
    frequency of occurrence = Daily,
    open issues = {},
}
