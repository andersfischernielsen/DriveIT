\section{Nonfunctional Requirements}
\subsection{Usability}
The \texttt{DriveIT Windows Client} must use the \textit{Design Guidelines of Microsoft\footnote{\url{http://msdn.microsoft.com/library/windows/apps/hh781237.aspx}}} such that it is intuitive for users who are used to the Windows platform.

Furthermore the UI must be minimalistic and the most frequently used features should be placed in view no further than two mouse clicks away from the start-up page. Some training of employees may be required prior to using the client.\\

The \texttt{DriveIT Web Client} should follow modern mobile-first website design thus making the site intuitive and easy to use for everyday users. Furthermore simplicity should be paramount, and viewing information about cars should be quick and linear (e.g. not interrupt the users current task).

\subsection{Reliability}
For the \texttt{DriveIT Web Client} and \texttt{DriveIT Web API} we use \textit{Microsoft Azure} to deploy the system. Due to this, we are not responsible for server maintenance and down time. \\
\textit{Microsoft Azure} is a reliable and fast deployment environment, saving development from the troubles of setting up and hosting a deployment and server environment.

The three systems should have sufficient exception handling and not crash the entire system, but instead provide an error message.

\subsection{Performance}
The system must be responsive and fast to use. Most user actions must not freeze the user interface, therefore giving the UI a responsive feel. The start-up time for the \texttt{DriveIT Windows Client} must be less than 3 seconds for a modern computer.

Furthermore the \texttt{DriveIT Web Client} must be fast to use and all pages should be loaded in less than 2 seconds for a 10mbit/1mbit connection excluding browser rendering time, for a modern browser.

\subsection{Maintainability}
The systems must be easy to maintain and extend which must be achieved through the use of good code structure using interfaces, design patterns, and extensive documentation. Changes in the backend of the \texttt{DriveIT System} requires a reinstall of the \texttt{DriveIT Web Client} and \texttt{Windows Client}.

\subsection{Portability}
The functionality of the \texttt{DriveIT} clients should be easy to port to other clients. This must be achieved by implementing the data access of the system using an open Web API with documented functionality, therefore allowing other clients to access the information. The \texttt{DriveIT Web Client} should be usable on mobile devices too.

\subsection{Implementation}
The different parts of the system must be implemented in different ways. It must have a main \texttt{DriveIT Web API} that must take care of persistence and communication between the \texttt{DriveIT Windows Client} and the \texttt{DriveIT Web Client}. The system must mainly be implemented using the programming language C$\sharp$. Implementation in a popular language such as C$\sharp$ assures supportability and extensibility. 
Using the \textit{.NET framework} offers a lot of developer tools making development more efficient.

\subsection{Operations}
The system should mainly be maintained by the developers of the \texttt{DriveIT System}. Training is required for employees so that they are able to use the system on a daily basis and solve minor issues that might occur on a day to day basis.
