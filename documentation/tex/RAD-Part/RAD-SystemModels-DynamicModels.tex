\section{Dynamic Models}

\subsection{Use Case Sequence Diagrams}
\begin{figure}[H]
	\centering
		\includegraphics[width=\textwidth]{Figures/SequenceDiagram-CreateSale}\\
		% place the figure in the Figures folder (located with the main file)
		% you need to fix the scale a few times to get it right, but latex does not compress so one can always zoom in to see details.
	\caption{Sequence diagram of the CreateSale use case.}
  \label{fig:SequenceDiagram-CreateSale}
  % label it something meanfull
\end{figure}
\\
This sequence diagram is made from the use case CreateSale. In this use case, the employee starts with opening the windows application, and thereafter navigate to the sale window. In the sale window, the employee fills in the information necessary to create a sale, and after the employee has finished filling in the information, he or she clicks on the "Create" button and the sale will then be created and saved in the persistent storage. The employee will then be prompted with a popup window, telling if the save was a success or an error has occured.
\\
\begin{figure}[H]
	\centering
		\includegraphics[width=\textwidth]{Figures/SequenceDiagram-CreateUserAccount}\\
		% place the figure in the Figures folder (located with the main file)
		% you need to fix the scale a few times to get it right, but latex does not compress so one can always zoom in to see details.
	\caption{Sequence diagram of the CreateUserAccount use case.}
  \label{fig:SequenceDiagram-CreateUserAccount}
  % label it something meanfull
\end{figure}
\\
This sequence diagram is made from the use case CreateUserAccount. In this use case, the non-registered customer has navigated to our web client and thereafter click "Register", whereafter the customer is directed to the CreateUserWindow. In this window, the user will be filling in the necessary information, and after the customer has finished filling in the information, he or she click on the "Create" button and the user account will then be created and saved in the persistent storage. The CreateUserWindow will then show that the creation of the user accout has been a success, and the customer can then log in into the web client.
\\
\begin{figure}[H]
	\centering
		\includegraphics[width=\textwidth]{Figures/SequenceDiagram-ContactInterestedCustomer}\\
		% place the figure in the Figures folder (located with the main file)
		% you need to fix the scale a few times to get it right, but latex does not compress so one can always zoom in to see details.
	\caption{Sequence diagram of the ContactInterestedCustomer use case.}
  \label{fig:SequenceDiagram-ContactInterestedCustomer}
  % label it something meanfull
\end{figure}
\\
This sequence diagram is made from the use case ContactInterestedCustomer. In this case, the employee starts with opening the windows application. The employee then navigates to the ContactRequestWindow, where there will be shown a list of contact requests by different customers. The employee picks one of the contact requests and a more detailed view of the contact request will be shown. The employee then calls the customer, starts a conversation regarding the contact request, and after a while the customer will confirm his or her choice and terminate the conversation. The employee will then close the contact request detail window and create a sale.
\\
\begin{figure}[H]
	\centering
		\includegraphics[width=\textwidth]{Figures/SequenceDiagram-RequestEmployeeContact}\\
		% place the figure in the Figures folder (located with the main file)
		% you need to fix the scale a few times to get it right, but latex does not compress so one can always zoom in to see details.
	\caption{Sequence diagram of the RequestEmployeeContact use case.}
  \label{fig:SequenceDiagram-RequestEmployeeContact}
  % label it something meanfull
\end{figure}
\\

\\
\subsection{State Diagrams}
\begin{figure}[H]
	\centering
		\includegraphics[width=\textwidth]{Figures/StateDiagram-Car}\\
		% place the figure in the Figures folder (located with the main file)
		% you need to fix the scale a few times to get it right, but latex does not compress so one can always zoom in to see details.
	\caption{State diagram of the Car object}
  \label{fig:StateDiagram-Car}
\end{figure}
The \ref{fig:StateDiagram-Car} shows the states a car entity can be in. The car should always be possible to delete as shown in the diagram, but to be sold the car must first enter the \emph{For Sale} state. In the \emph{For Sale} and \emph{Sold} state the car should be visible from the \texttt{DriveIT Web Client} with a label indicating which of the two states its in, and the car should be visible from the \texttt{DriveIT Windows Client} in all its states.\\
\begin{figure}[H]
	\centering
		\includegraphics[width=\textwidth]{Figures/StateDiagram-ContactRequest}\\
		% place the figure in the Figures folder (located with the main file)
		% you need to fix the scale a few times to get it right, but latex does not compress so one can always zoom in to see details.
	\caption{State diagram of the ContactRequest object}
  \label{fig:StateDiagram-ContactRequest}
  % label it something meanfull
\end{figure}