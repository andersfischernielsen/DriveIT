\section{Dynamic Models}

\subsection{Use Case Sequence Diagrams}
\begin{figure}[H]
	\centering
		\includegraphics[width=\textwidth]{Figures/SequenceDiagram-CreateOrder}\\
		% place the figure in the Figures folder (located with the main file)
		% you need to fix the scale a few times to get it right, but latex does not compress so one can always zoom in to see details.
	\caption{Sequence diagram of the CreateOrder use case.}
  \label{fig:SequenceDiagram-CreateOrder}
  % label it something meanfull
\end{figure}
\begin{figure}[H]
	\centering
		\includegraphics[width=\textwidth]{Figures/SequenceDiagram-CreateUserAccount}\\
		% place the figure in the Figures folder (located with the main file)
		% you need to fix the scale a few times to get it right, but latex does not compress so one can always zoom in to see details.
	\caption{Sequence diagram of the CreateUserAccount use case.}
  \label{fig:SequenceDiagram-CreateUserAccount}
  % label it something meanfull
\end{figure}
\begin{figure}[H]
	\centering
		\includegraphics[width=\textwidth]{Figures/SequenceDiagram-ContactInterestedCustomer}\\
		% place the figure in the Figures folder (located with the main file)
		% you need to fix the scale a few times to get it right, but latex does not compress so one can always zoom in to see details.
	\caption{Sequence diagram of the ContactInterestedCustomer use case.}
  \label{fig:SequenceDiagram-ContactInterestedCustomer}
  % label it something meanfull
\end{figure}
\begin{figure}[H]
	\centering
		\includegraphics[width=\textwidth]{Figures/SequenceDiagram-RequestEmployeeContact}\\
		% place the figure in the Figures folder (located with the main file)
		% you need to fix the scale a few times to get it right, but latex does not compress so one can always zoom in to see details.
	\caption{Sequence diagram of the RequestEmployeeContact use case.}
  \label{fig:SequenceDiagram-RequestEmployeeContact}
  % label it something meanfull
\end{figure}

\subsection{State Diagrams}
\begin{figure}[H]
	\centering
		\includegraphics[width=\textwidth]{Figures/StateDiagram-Car}\\
		% place the figure in the Figures folder (located with the main file)
		% you need to fix the scale a few times to get it right, but latex does not compress so one can always zoom in to see details.
	\caption{State diagram of the Car object}
  \label{fig:StateDiagram-Car}
\end{figure}
The \ref{fig:StateDiagram-Car} shows the states a car entity can be in. The car should always be possible to delete as shown in the diagram, but to be sold the car must first enter the \emph{For Sale} state. In the \emph{For Sale} and \emph{Sold} state the car should be visible from the \texttt{DriveIT Web Client} with a label indicating which of the two states its in, and the car should be visible from the \texttt{DriveIT Windows Client} in all its states.\\
\begin{figure}[H]
	\centering
		\includegraphics[width=\textwidth]{Figures/StateDiagram-ContactRequest}\\
		% place the figure in the Figures folder (located with the main file)
		% you need to fix the scale a few times to get it right, but latex does not compress so one can always zoom in to see details.
	\caption{State diagram of the ContactRequest object}
  \label{fig:StateDiagram-ContactRequest}
  % label it something meanfull
\end{figure}