\section{Domain Object Models}
\begin{figure}[h!]
	\centering
		\includegraphics[scale=0.35]{Figures/DomainObjectModel}\\
		% place the figure in the Figures folder (located with the main file)
		% you need to fix the scale a few times to get it right, but latex does not compress so one can always zoom in to see details.
	\caption{\textbf{Domain Object Model.}}
  \label{fig:DomainObjectModel}
  % label it something meanfull
\end{figure}
\textbf{Figure ~\ref{fig:DomainObjectModel}}
The \texttt{Car} object here is a given car the car lot has purchased. A \texttt{Car} can have many or no \texttt{Comment}s. A  \texttt{Comment} has only one  \texttt{Customer}, but one  \texttt{Customer} can create many  \texttt{Comment}s.\\
A \texttt{Customer} can create many \texttt{Contact Request}s, but a given \texttt{Contact Request} can only come from one \texttt{Customer}.\\
When a  \texttt{Car} is purchased a \texttt{Sale} is created. A \texttt{Sale} can only have one \texttt{Customer}, one \texttt{Car} and one \texttt{Employee}.\\
One \texttt{Customer} can have many sales (if he/she buys many cars), but a given \texttt{Car} can only be sold once. An \texttt{Employee} can likewise have sold many cars.\\
The \texttt{Admin} inherits from the \texttt{Employee}.