\section{Global Software Control}

The flow of the \texttt{DriveIT System} would be a \textit{Procedure-driven control}\footnote{{Object-Oriented Sofware Engineering - 3rd Edition - 2009 - page 275}}, and the reason for this, is that the program will follow specific procedures, depending on how the \texttt{Customer}/\texttt{Employee} is interacting with the system. An exception to the flow model is the \texttt{DriveIT Windows Client}. This subsystem uses an event-driven control flow for the data-binding between the \texttt{View} and \texttt{ViewModel}s.\\

By using the \textit{Procedure-driven control} approach, some concurrency and synchronisation problems may occur. If two \texttt{Employee}s update the same entity which were read at the same time, then the last update will override the first. Due to the scope of the project, such problems are not handled in the \texttt{Client}s.\\

Upon starting the system, the \texttt{Customer} will be presented with the main view of the system. It is not required to log into the system to use it. The \texttt{Customer} can freely navigate through the database searching for cars given the search criteria. Only when the \texttt{Customer} wishes to be contacted about a \texttt{Car}, a login will be prompted.\\ 

This differs a bit for the \texttt{Employee} as their usage of the system is different. An \texttt{Employee} has tasks which require a higher access level, that are not available for a \texttt{Customer}. Therefore an \texttt{Employee} will log in upon starting up the system, and hereby verify the authorization level.\\

For logging in, the \texttt{LoginViewModel} of our \texttt{ViewModel Subsystem} will use the information entered into our login view to authenticate the information, and if the authentication through the \texttt{EmployeeController} of the \texttt{Controller Subsystem} is a success, the \texttt{Persistence Subsystem} will access the information belonging to the \texttt{Customer}/\texttt{Employee}, and our \texttt{ViewModel Subsystem} will then observe the \texttt{Persistence Subsystem} and update its state, whereafter the \texttt{DriveITMainView}, from our \texttt{View Subsystem}, will be initialized and show the data received from the database.\\

The \texttt{Customer} can now interact with the system in multiple ways. For instance, the \texttt{Customer} can view the cars they are interested in. This task is handled by the \texttt{Controller Subsystem}, which updates the state of the \texttt{Model Subsystem} which is then reflected in the \texttt{DriveITMainView}, which will register the activity of the \texttt{Customer}, update the state of the Model and then load it into the DriveITSystemView. In the background, new activity will be stored in the database, as the \texttt{Customer} is still using the system. If a change is commited, the Model will update its own state, whereafter the \texttt{DriveITMainView} will be updated by the \texttt{ViewModel Subsystem}.\\

On the other hand an \texttt{Employee} will have the main task of handling requests for contact by a \texttt{Customer}. Anybody with \texttt{Employee} authentication, will have a shared `forum', where they can view a list of \texttt{Users} who are interested in cars/contact from an \texttt{Employee}. An \texttt{Employee} can now drag one of these requests into their own workfield and hereby claim a task. The same way as the task for a \texttt{Customer} above, the task will be  handled by the \texttt{Controller Subsystem}, which updates the state of the \texttt{ViewModel Subsystem} which is then reflected in the \texttt{DriveITMainView}. This will register the activity of the \texttt{Employee}, update the state of the Model and then load it into the \texttt{DriveITMainView}. In the background, new activity will be stored in the database, as the \texttt{Employee} is still using the system. If a change is commited, the Model will update its own state, whereafter the \texttt{DriveITMainView} will be updated by the  \texttt{ViewController Subsystem}.\\

