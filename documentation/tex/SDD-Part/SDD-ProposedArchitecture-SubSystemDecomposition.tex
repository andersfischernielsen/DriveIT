\section{Subsystem Decomposition}
We have defined the following subsystems:
\begin{itemize}
	\item Persistence subsystem: A Microsoft Database used for storing the data from the applications.
	\item WebAPI: The WebAPI is the only way to communicate with the persistence subsystem. This means that every table in the persistence module must be supported by the API. 
	The API is built out of two main halves; a module for storing and retrieving Entities in a hosted database using Entity Framework and modules for serialising the Entities into JSON (or XML) and transferring these via a REST interface. These modules are implemented using the MVC framework of .NET. 
	The API furthermore uses Authorization to decide if a given user has access to certain parts of the persistence module.
	\item WindowsApp: Used by the employees to serve the customers. The Windows application can only be accessed by employees who will have to login at startup. The application is used to CRUD cars, orders, customers and employees (by the administrator account-type).
	\item CarQuery: The CarQuery subsystem is a collection of classes that enable the system to fill out missing information about cars from the CarQuery API. \\
	This is used by the employees when creating a new car. The subsystem consists of a JSON deserialiser and a class that fills out the missing attributes of a Car object from the deserialised JSON data. 
	The employee can fill out the information he/she knows about the car and the system will then narrow down its search on the CarQuery API. 
\end{itemize}