\section{Design Goals}
\subsection{Powerful and Easy Usability}
The goal with the usability of the \texttt{DriveIT Windows Client} is to give employees a powerful and easy to use tool to perform their work tasks. The \textit{Gestalt Principles}\footnote{\url{http://en.wikipedia.org/wiki/Gestalt_psychology}} are excellent examples on how to make a clean and simple user interface. By using the \textit{Gestalt Principles}, we are able to make a clear distinction of which parts of the user interface that belongs to each other. This makes it easier for the user to interact with the user interface. To give the \texttt{Employee}s a powerful tool, the most frequently used tasks must be achievable quickly, but with possible extensions to allow more advanced options. This is making it fast to use the interface, while still maintaining enough complexity to be in appliance with the requirements.\\

For the usability of the \texttt{DriveIT Web Client} the goal is to have a website which is simple and provides the \texttt{Customer} with only a few actions. This will decrease the possibility of confusion and therefore hopefully keep the users on the site. \textit{Gestalt Principles} should be used in the making of the \textit{User Interface} (henceforth UI) such that it is easy for the user to see which functions map together.

\subsection{High Reliability} It is desired to create a reliable program that will prevent the user from losing progress made in the system due to unexpected events e.g. a crash. By synchronising with the \textit{Microsoft Azure} server after every \textit{CRUD}\footnote{\url{http://en.wikipedia.org/wiki/Create,_read,_update_and_delete}} action and not just at start up and shut down, we ensure that a minimum amount of data is lost, if a crash should occur. Should the \texttt{DriveIT Windows Client} loose connection to the server a message must be shown to the user. All CRUD actions will fail until a connection is re established. This is a design choice which is chosen due to the time limit of the project. \\
Force restarting is not acceptable and most exceptions must be caught and handled during runtime. By doing the above mentioned, data loss will be kept to a minimum by limiting it to the user's current activity, should a failure occur. Due to time constraints the exception handling of both \texttt{DriveIT Clients} are not implemented in such a manner, that the error message presents enough details about the error that occurred.

\subsection{Strong Architecture with Focus on Extensibility}
When rolling out future updates for the \texttt{DriveIT Systems}, a full re-installation will be necessary. This is due to time constraints and because this feature is not part of the scope. It would be desirable to have a patch system, since this makes it less resource demanding to update the system in the future. It would also make it easier for the \texttt{Employee}s to just accept and update instead of manually having to uninstall the application and then reinstall it.\\

\subsection{Good Documentation and Testing of the Most Important Subsystems}
The \texttt{DriveIT System}s must be tested before release, but in a limited way, as a full unit, integration, and acceptance test are out of the scope due to the time frame set for the development of the system. Focus has therefore been on testing key components which are central and critical parts of our system, while also working as a template for future tests.\\

Documentation is a central part of the \texttt{DriveIT System}, as it is important to document how the system works as a whole and also how the modules and subsystems work separately. By making this documentation it will be easier to extend the system by developers that have no knowledge about the \texttt{DriveIT System}.\\

\subsection{High Portability}
To make sure that the system is easy to extend and scale it will be built around a web API which makes it easy to communicate with the system. This allows other developers to make applications for other platforms i.e. mobile devices by using the \texttt{DriveIT Web API}.\\

\subsection{Easy and Efficient Implementation}
The fact that the system will be developed and maintained in a popular and broadly used programming language, C$\sharp$, assures a relatively painless implementation process. Since C$\sharp$ uses the .NET platform there are a lot of well designed and highly maintained tools available, minimizing the development time and increasing the quality of the software.\\

\subsection{Security as a Secondary Priority}
It is not a part of the scope of this project to take security into consideration. However security has to some extent been considered, since authorization is required to access to certain parts of the \texttt{DriveIT System}. Using the \textit{ASP.NET Identity}\footnote{\url{http://www.asp.net/identity}} framework, passwords are hashed when they reach the \texttt{DriveIT Web API}, though not until then. Because \textit{SSL}\footnote{\url{http://en.wikipedia.org/wiki/Secure_Sockets_Layer}} is out of scope data sent between the \texttt{DriveIT System}s are not encrypted.
