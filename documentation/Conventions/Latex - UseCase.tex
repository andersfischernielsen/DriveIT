%Template for adding a usecase to the documentation
% BRUG , I STEDET FOR ITEM MELLEM HVERT PUNKT!
% \usepackage{usecases}
\usecase{
    title = {ViewTrafficInfo},
    label = uc:view_traffic_info,
    description = {The \emph{Driver} can see the traffic information on the \texttt{InformationSign}.},
%    scope = {},
%    level = {},
    actors = {Initiated by \emph{Driver}},
%    stakeholders and interests = {},
    precondition = {The \texttt{InformationSign} is initialized and online.},
%    preconditions = {
%        \item 1
%        \item 2
%        \item 3  
%    },
    postcondition = {The \emph{Driver} has received traffic information.},
%    postconditions = {
%        \item 1
%        \item 2
%        \item 3
%    },
    main success scenario = {
        The \texttt{InformationSign} is updated with relevant traffic information.,
        The \emph{Driver} sees traffic information.
    },
%    extensions = {
%        \item 1
%        \item 2
%        \item 3
%    },
%    special requirements = {
%	    \item 1
%        \item 2
%		\\
%	},
    frequency of occurrence = Often,
    open issues = {It is quite unclear if a better name for this use case would rather be `DisplayTrafficInfo'?},
}